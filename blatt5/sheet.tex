% Graphenalgorithmen
% Lukas Kalbertodt
% Mirco Wagner
% Elena Resch
% ===== Header ===============================================================
\documentclass[11pt]{scrartcl} % europäische Artikel Klasse
\usepackage[top=3cm, bottom=4.5cm, left=3.25cm, right=3.25cm]{geometry}
% Pakete für deutschen Text (Umlaute) + Font
\usepackage[utf8]{inputenc}
\usepackage[T1]{fontenc}
\usepackage{lmodern}
\usepackage{ngerman}
\usepackage[ngerman]{babel} % Deutsche Anführungszeichen
\usepackage{amsmath}
\usepackage{amsfonts} % \mathbb
\usepackage{relsize}
\usepackage{csquotes} % \enquote
\usepackage{paralist} % Beliebige Aufzählungszeichen
\usepackage{tabularx} % Tolle Tabellen
\usepackage{longtable} % Tolle Tabellen
\usepackage{caption}
\usepackage[labelformat=empty]{subcaption}

% Deutsche Absätze: Kein Einrücken, aber Abstand
\parskip=10pt
\parindent=0pt

\usepackage{xcolor}
\definecolor{darkblue}{rgb}{0,0,0.4}


\usepackage{tikz} % Graphen zeichnen
\usetikzlibrary{arrows}
\usetikzlibrary{positioning}
\newcommand{\func}[1]{\mbox{\emph{#1}}}

% Dokument-Metadaten
\title{Graphenalgorithmen: Blatt 5}
\author{Lukas Kalbertodt, Elena Resch, Mirco Wagner}

% Tikz style definitions
% vertex in spanning tree
\tikzstyle{vInSp}=[draw=black, double=black,shape=circle, minimum size=20pt,inner sep=0pt]
% vertex not in spanning tree
\tikzstyle{vNotSp}=[draw=black!25,shape=circle,text=black!60, minimum size=20pt,inner sep=0pt]
\tikzstyle{eInSp} = [draw=darkblue!75,thick,-]  % edge in spanning tree
\tikzstyle{eNotSp} = [draw=black!25,-]    % edge not in spanning tree
\tikzstyle{eNewSp} = [draw=red,very thick,-]    % edge new to spanning tree
\tikzstyle{weight} = [draw=none,shape=circle, fill=white,inner sep=1pt,font=\scriptsize]

% ===== Body =================================================================
\begin{document}
\maketitle


\section*{Aufgabe 10:}
\emph{Hinweis}: Es wird angenommen, dass mit \enquote{Heap} ein \enquote{BinaryHeap} gemeint ist.\\[0.2cm]
\emph{Legende für den Graphen}: Dick schwarz umkreiste Knoten sind im Spannbaum enthalten. Die dicke rote Kante wird in dem Schritt als neue Kante dem Spannbaum hinzugefügt. Blaue Kanten befinden sich bereits im Spannbaum.\\[0.2cm]
\emph{Legende for den BinaryHeap}:

% shortcut for seperation within the table between steps
\newcommand{\stepsep}{\\ \multicolumn{2}{c}{\textcolor{lightgray}{\rule{3.5cm}{0.4pt}}} \\}

\begin{longtable}{p{0.45\textwidth} p{0.45\textwidth}}
\begin{tikzpicture}[scale=1.4]
    \foreach \pos /\name /\style in
    {{(0,1)}/a/vInSp,{(1,2)}/b/vNotSp,{(1,0)}/c/vNotSp,{(2,1)}/d/vNotSp,{(3,2)}/e/vNotSp,{(3,0)}/f/vNotSp}
        \node[\style](\name) at \pos{$\name$};
    \foreach \source /\dest /\weight /\style in
    {a/b/4/eNotSp, a/c/2/eNewSp, b/c/5/eNotSp, b/d/9/eNotSp, b/e/6/eNotSp, c/d/8/eNotSp, c/f/7/eNotSp, d/e/3/eNotSp, d/f/10/eNotSp, e/f/9/eNotSp}
        \path[\style] (\source) -- node[weight] {$\weight$} (\dest);
\end{tikzpicture}
& % -----------------------
\stepsep % ==================================
\begin{tikzpicture}[scale=1.4]
    \foreach \pos /\name /\style in
    {{(0,1)}/a/vInSp,{(1,2)}/b/vNotSp,{(1,0)}/c/vInSp,{(2,1)}/d/vNotSp,{(3,2)}/e/vNotSp,{(3,0)}/f/vNotSp}
        \node[\style](\name) at \pos{$\name$};
    \foreach \source /\dest /\weight /\style in
    {a/b/4/eNewSp, a/c/2/eInSp, b/c/5/eNotSp, b/d/9/eNotSp, b/e/6/eNotSp, c/d/8/eNotSp, c/f/7/eNotSp, d/e/3/eNotSp, d/f/10/eNotSp, e/f/9/eNotSp}
        \path[\style] (\source) -- node[weight] {$\weight$} (\dest);
\end{tikzpicture}
& % -----------------------
\stepsep % ==================================
\begin{tikzpicture}[scale=1.4]
    \foreach \pos /\name /\style in
    {{(0,1)}/a/vInSp,{(1,2)}/b/vInSp,{(1,0)}/c/vInSp,{(2,1)}/d/vNotSp,{(3,2)}/e/vNotSp,{(3,0)}/f/vNotSp}
        \node[\style](\name) at \pos{$\name$};
    \foreach \source /\dest /\weight /\style in
    {a/b/4/eInSp, a/c/2/eInSp, b/c/5/eNotSp, b/d/9/eNotSp, b/e/6/eNewSp, c/d/8/eNotSp, c/f/7/eNotSp, d/e/3/eNotSp, d/f/10/eNotSp, e/f/9/eNotSp}
        \path[\style] (\source) -- node[weight] {$\weight$} (\dest);
\end{tikzpicture}
& % -----------------------
\stepsep % ==================================
\begin{tikzpicture}[scale=1.4]
    \foreach \pos /\name /\style in
    {{(0,1)}/a/vInSp,{(1,2)}/b/vInSp,{(1,0)}/c/vInSp,{(2,1)}/d/vNotSp,{(3,2)}/e/vInSp,{(3,0)}/f/vNotSp}
        \node[\style](\name) at \pos{$\name$};
    \foreach \source /\dest /\weight /\style in
    {a/b/4/eInSp, a/c/2/eInSp, b/c/5/eNotSp, b/d/9/eNotSp, b/e/6/eInSp, c/d/8/eNotSp, c/f/7/eNotSp, d/e/3/eNewSp, d/f/10/eNotSp, e/f/9/eNotSp}
        \path[\style] (\source) -- node[weight] {$\weight$} (\dest);
\end{tikzpicture}
& % -----------------------
\stepsep % ==================================
\begin{tikzpicture}[scale=1.4]
    \foreach \pos /\name /\style in
    {{(0,1)}/a/vInSp,{(1,2)}/b/vInSp,{(1,0)}/c/vInSp,{(2,1)}/d/vInSp,{(3,2)}/e/vInSp,{(3,0)}/f/vNotSp}
        \node[\style](\name) at \pos{$\name$};
    \foreach \source /\dest /\weight /\style in
    {a/b/4/eInSp, a/c/2/eInSp, b/c/5/eNotSp, b/d/9/eNotSp, b/e/6/eInSp, c/d/8/eNotSp, c/f/7/eNewSp, d/e/3/eInSp, d/f/10/eNotSp, e/f/9/eNotSp}
        \path[\style] (\source) -- node[weight] {$\weight$} (\dest);
\end{tikzpicture}
& % -----------------------
\stepsep % ==================================
\begin{tikzpicture}[scale=1.4]
    \foreach \pos /\name /\style in
    {{(0,1)}/a/vInSp,{(1,2)}/b/vInSp,{(1,0)}/c/vInSp,{(2,1)}/d/vInSp,{(3,2)}/e/vInSp,{(3,0)}/f/vInSp}
        \node[\style](\name) at \pos{$\name$};
    \foreach \source /\dest /\weight /\style in
    {a/b/4/eInSp, a/c/2/eInSp, b/c/5/eNotSp, b/d/9/eNotSp, b/e/6/eInSp, c/d/8/eNotSp, c/f/7/eInSp, d/e/3/eInSp, d/f/10/eNotSp, e/f/9/eNotSp}
        \path[\style] (\source) -- node[weight] {$\weight$} (\dest);
\end{tikzpicture}
& % -----------------------
\end{longtable}

\end{document}

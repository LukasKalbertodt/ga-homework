% Graphenalgorithmen
% Lukas Kalbertodt
% Mirco Wagner
% Elena Resch
% ===== Header ===============================================================
\documentclass[11pt]{scrartcl} % europäische Artikel Klasse
\usepackage[top=3cm, bottom=4.5cm, left=3.25cm, right=3.25cm]{geometry}
% Pakete für deutschen Text (Umlaute) + Font
\usepackage[utf8]{inputenc}
\usepackage[T1]{fontenc}
\usepackage{lmodern}
\usepackage{ngerman}
\usepackage[ngerman]{babel} % Deutsche Anführungszeichen
\usepackage{amsmath}
\usepackage{amsfonts} % \mathbb
\usepackage{relsize}
\usepackage{csquotes} % \enquote
\usepackage{paralist} % Beliebige Aufzählungszeichen
\usepackage{tabularx} % Tolle Tabellen
\usepackage{longtable} % Tolle Tabellen
\usepackage{caption}
\usepackage[labelformat=empty]{subcaption}
\usepackage{etoolbox}
\usepackage{xifthen}

% Deutsche Absätze: Kein Einrücken, aber Abstand
\parskip=10pt
\parindent=0pt

\usepackage{xcolor}
\definecolor{darkblue}{rgb}{0,0,0.4}


\usepackage{tikz} % Graphen zeichnen
\usetikzlibrary{arrows}
\usetikzlibrary{positioning}
\newcommand{\func}[1]{\mbox{\texttt{#1}}}

% Dokument-Metadaten
\title{Graphenalgorithmen: Blatt 5}
\author{Lukas Kalbertodt, Elena Resch, Mirko Wagner}

% Tikz style definitions
% vertex in spanning tree
\tikzstyle{vInSp}=[draw=black, double=black,shape=circle, minimum size=20pt,inner sep=0pt]
% vertex not in spanning tree
\tikzstyle{vNotSp}=[draw=black!25,shape=circle,text=black!60, minimum size=20pt,inner sep=0pt]
\tikzstyle{eInSp} = [draw=darkblue!75, very thick,-]  % edge in spanning tree
\tikzstyle{eNotSp} = [draw=black!25,-]    % edge not in spanning tree
\tikzstyle{eNewSp} = [draw=red,ultra thick,-]    % edge new to spanning tree
\tikzstyle{weight} = [draw=none,shape=circle, fill=white,inner sep=1pt,font=\scriptsize]
\tikzstyle{vHeap} = [draw=black,shape=circle,inner sep=2pt] % vertex in heap
\tikzstyle{eHeap} = [draw=black,thick,<-]  % edge in heap

%%% Some commands for outputting stuff
% To draw a heap. Argument looks like that: {(0,1)}/b/{a,4}/x,{(1,2)}/c/{a,2}/b
% List of (position)/name/pred-cost/father ... When no father -> father=x
% Call: \drawheap{{(0,1)}/b/{a,4}/x,{(1,2)}/c/{a,2}/b}
\newcommand{\drawheap}[1] {
    \newline
    \begin{tikzpicture}[scale=1.4]
        \foreach \pos /\name /\predcost /\father in {#1} {
            \node[vHeap](\name) [align=center] at \pos{\small{$\name$}\\[-0.3cm] \tiny{\predcost} };
            \ifthenelse{\NOT \equal{\father}{x}} {
                \path[eHeap](\father) -> (\name);
            } {};
        }
    \end{tikzpicture}
}
% To draw the graph. Nodes as first argument, edges as second argument.
% Nodes: (pos)/name/style   (see below for example)
% Edges: source/dest/cost/style   (see below for example)
\newcommand{\drawgraph}[2] {
    \newline
    \begin{tikzpicture}[scale=1.4]
        \foreach \pos /\name /\style in {#1}
            \node[\style](\name) at \pos{$\name$};
        \foreach \source /\dest /\weight /\style in {#2}
            \path[\style] (\source) -- node[weight] {$\weight$} (\dest);
    \end{tikzpicture}
}

% To print a list of heap operations
\newcommand{\printheapops}[1]{
    \newline \small \begin{itemize}
        #1
    \end{itemize}
}

% shortcut for seperation within the table between steps
\newcommand{\stepsep}{\\ \multicolumn{3}{c}{\textcolor{lightgray}{\rule{8cm}{0.4pt}}} \\}

% ===== Body =================================================================
\begin{document}
\maketitle


\section*{Aufgabe 10:}
\emph{Hinweis}: Es wird angenommen, dass mit \enquote{Heap} ein \enquote{BinaryHeap} gemeint ist.\\[0.2cm]
\emph{Legende für den Graphen}: Dick schwarz umkreiste Knoten sind im Spannbaum enthalten. Die dicke rote Kante wird in dem Schritt als neue Kante dem Spannbaum hinzugefügt. Blaue Kanten befinden sich bereits im Spannbaum.\\[0.2cm]
\emph{Legende für den BinaryHeap}: Der dickere Buchstabe in jedem Knoten des Heaps steht für den Knoten im Graphen. In der zweiten Zeile steht zuerst der Predecessor und dann die Kosten.


\begin{longtable}{p{0.4\textwidth} p{0.23\textwidth} p{0.29\textwidth}}
\drawgraph
    {{(0,1)}/a/vInSp,{(1,2)}/b/vNotSp,{(1,0)}/c/vNotSp,{(2,1)}/d/vNotSp,{(3,2)}/e/vNotSp,{(3,0)}/f/vNotSp}
    {a/b/4/eNotSp, a/c/2/eNewSp, b/c/5/eNotSp, b/d/9/eNotSp, b/e/6/eNotSp, c/d/8/eNotSp, c/f/7/eNotSp, d/e/3/eNotSp, d/f/10/eNotSp, e/f/9/eNotSp}
& \drawheap{{(1,2)}/c/{a,2}/x,{(0,1)}/b/{a,4}/c}
& \printheapops{
    \item \func{insert(b, a, 4)}
    \item \func{insert(c, a, 2)}
}

\stepsep % ==================================
\drawgraph
    {{(0,1)}/a/vInSp,{(1,2)}/b/vNotSp,{(1,0)}/c/vInSp,{(2,1)}/d/vNotSp,{(3,2)}/e/vNotSp,{(3,0)}/f/vNotSp}
    {a/b/4/eNewSp, a/c/2/eInSp, b/c/5/eNotSp, b/d/9/eNotSp, b/e/6/eNotSp, c/d/8/eNotSp, c/f/7/eNotSp, d/e/3/eNotSp, d/f/10/eNotSp, e/f/9/eNotSp}
& \drawheap{{(1,2)}/b/{a,4}/x, {(0,1)}/d/{c,8}/b, {(2,1)}/f/{c,7}/b}
& \printheapops{
    \item \func{remove(c)}
    \item \func{insert(d, c, 8)}
    \item \func{insert(f, c, 7)}
}

\stepsep % ==================================
\drawgraph
    {{(0,1)}/a/vInSp,{(1,2)}/b/vInSp,{(1,0)}/c/vInSp,{(2,1)}/d/vNotSp,{(3,2)}/e/vNotSp,{(3,0)}/f/vNotSp}
    {a/b/4/eInSp, a/c/2/eInSp, b/c/5/eNotSp, b/d/9/eNotSp, b/e/6/eNewSp, c/d/8/eNotSp, c/f/7/eNotSp, d/e/3/eNotSp, d/f/10/eNotSp, e/f/9/eNotSp}
& \drawheap{{(1,2)}/e/{b,6}/x, {(0,1)}/d/{c,8}/b, {(2,1)}/f/{c,7}/b}
& \printheapops{
    \item \func{remove(b)}
    \item \func{insert(e, b, 6)}
}
\stepsep % ==================================
\drawgraph
    {{(0,1)}/a/vInSp,{(1,2)}/b/vInSp,{(1,0)}/c/vInSp,{(2,1)}/d/vNotSp,{(3,2)}/e/vInSp,{(3,0)}/f/vNotSp}
    {a/b/4/eInSp, a/c/2/eInSp, b/c/5/eNotSp, b/d/9/eNotSp, b/e/6/eInSp, c/d/8/eNotSp, c/f/7/eNotSp, d/e/3/eNewSp, d/f/10/eNotSp, e/f/9/eNotSp}
& \drawheap{{(1,2)}/d/{e,3}/x, {(0,1)}/f/{c,7}/b}
& \printheapops{
    \item \func{remove(e)}
    \item \func{update(d, e, 3)}
}
\stepsep % ==================================
\drawgraph
    {{(0,1)}/a/vInSp,{(1,2)}/b/vInSp,{(1,0)}/c/vInSp,{(2,1)}/d/vInSp,{(3,2)}/e/vInSp,{(3,0)}/f/vNotSp}
    {a/b/4/eInSp, a/c/2/eInSp, b/c/5/eNotSp, b/d/9/eNotSp, b/e/6/eInSp, c/d/8/eNotSp, c/f/7/eNewSp, d/e/3/eInSp, d/f/10/eNotSp, e/f/9/eNotSp}
& \drawheap{{(1,2)}/f/{c,7}/x}
& \printheapops{
    \item \func{remove(d)}
}
\stepsep % ==================================
\drawgraph
    {{(0,1)}/a/vInSp,{(1,2)}/b/vInSp,{(1,0)}/c/vInSp,{(2,1)}/d/vInSp,{(3,2)}/e/vInSp,{(3,0)}/f/vInSp}
    {a/b/4/eInSp, a/c/2/eInSp, b/c/5/eNotSp, b/d/9/eNotSp, b/e/6/eInSp, c/d/8/eNotSp, c/f/7/eInSp, d/e/3/eInSp, d/f/10/eNotSp, e/f/9/eNotSp}
&
& \printheapops{
    \item \func{remove(f)}
}
\end{longtable}
Kosten des MST: $c= 22$
\newpage
\section*{Aufgabe 11:}
\begin{compactenum}[(a)]
    \item Seien alle Kantengewichte in $G$ paarweise unterschiedlich. Gegeben sei $T$, ein minimaler Spannbaum von $G$ mit dem Wert $v$. Angenommen es gibt einen weiteren Spannbaum $T^* \ne T$ von $G$ mit dem Wert $v^*$, wobei $v^* = v$ (also ist $T^*$ auch minimal).\\
    Wir wählen nun eine Kante $e \in T \setminus T^*$ und löschen sie aus $T$. $T$ ist jetzt ein Wald mit zwei Komponenten/Bäumen. $T^*$ besitzt eine Kante $e^*$, die diese beiden Komponenten in $T$ verbindet. Nach Vorraussetzung ist $c(e) \ne c(e^*)$, somit können genau zwei Fälle auftreten:\\

    \begin{compactenum}[1.]
        \item $c(e) > c(e^*)$: Wenn wir in $T$ die Kante $e$ durch $e*$ ersetzen würden, hätten wir wieder einen Spannbaum, aber mit geringeren Kosten. Weder $T$ noch $T^*$ wären also minimale Spannbäume: Wiederspruch!
        \item $c(e) < c(e^*)$: Analog.
    \end{compactenum}
    \hfill $\square$\\

    \item Die Behauptung ist falsch. Im folgenden trivialen Gegenbeispiel ist der Spannbaum eindeutig bestimmt, aber es gibt zwei gleiche Kantenkosten:\\
    \begin{center}
    \begin{tikzpicture}[-,auto,node distance=2cm, thick,node/.style={circle,draw,inner sep=0pt, minimum size=20pt}]
        \node[node] (a) {};
        \node[node] (b) [right of=a] {};
        \node[node] (c) [right of=b] {};
        \path[every node/.style={font=\sffamily\small}]
            (a) edge [] node[] {7} (b)
            (b) edge [] node[] {7} (c);
    \end{tikzpicture}\\[0.5cm]
    \end{center}

    \item In allen Graphen, die in zwei Komponenten zerfallen, wenn man die teuerste Kante löscht, ist die teuerste Kante offentsichtlich auch Teil jedes Spannbaums. Minimalbeispiel:\\
    \begin{center}
    \begin{tikzpicture}[-,auto,node distance=2cm, thick,node/.style={circle,draw,inner sep=0pt, minimum size=20pt}]
        \node[node] (a) {};
        \node[node] (b) [right of=a] {};
        \path[every node/.style={font=\sffamily\small}]
            (a) edge [] node[] {7} (b);
    \end{tikzpicture}
    \end{center}

\end{compactenum}
\end{document}

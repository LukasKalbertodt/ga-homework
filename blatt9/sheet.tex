% Graphenalgorithmen
% Lukas Kalbertodt
% Mirko Wagner
% Elena Resch
% ===== Header ===============================================================
\documentclass[11pt]{scrartcl} % europäische Artikel Klasse
\usepackage[top=3cm, bottom=4.5cm, left=3cm, right=3cm]{geometry}
% Pakete für deutschen Text (Umlaute) + Font
\usepackage[utf8]{inputenc}
\usepackage[T1]{fontenc}
\usepackage{lmodern}
\usepackage{ngerman}
\usepackage[ngerman]{babel} % Deutsche Anführungszeichen
\usepackage{amsmath}
\usepackage{amsfonts} % \mathbb
\usepackage{relsize}
\usepackage{csquotes} % \enquote
\usepackage{paralist} % Beliebige Aufzählungszeichen
\usepackage{tabularx} % Tolle Tabellen
\usepackage{caption}
\usepackage{wasysym}
\usepackage[labelformat=empty]{subcaption}
\usepackage{longtable}

% Deutsche Absätze: Kein Einrücken, aber Abstand
\parskip=10pt
\parindent=0pt

\usepackage{tikz} % Graphen zeichnen
\usetikzlibrary{arrows}
\usetikzlibrary{positioning}
\usetikzlibrary{decorations.pathmorphing}
\newcommand{\func}[1]{\mbox{\emph{#1}}}

% Dokument-Metadaten
\title{Graphenalgorithmen: Blatt 9}
\author{Lukas Kalbertodt, Elena Resch, Mirko Wagner}

% ===== Body =================================================================
\begin{document}
\maketitle
\section*{Aufgabe 18:}
\tikzstyle{node} = [circle,draw,inner sep=0pt, minimum size=20pt,font=\scriptsize]
\tikzstyle{edge} = [draw=black,thick]
\tikzstyle{weight} = [shape=circle, fill=white,inner sep=1pt,font=\scriptsize]

\begin{tikzpicture}[-]
    \node[node] (1) at(0,4) {1};
    \node[node] (2) at(2,5) {2};
    \node[node] (3) at(2,3) {3};
    \node[node] (4) at(2,0.5) {4};
    \node[node] (5) at(4,5) {5};
    \node[node] (6) at(4,3) {6};
    \node[node] (7) at(4,1.5) {7};
    \node[node] (8) at(4,0.5) {8};
    \node[node] (9) at(6,5) {9};
    \node[node] (10) at(6,3) {10};
    \node[node] (11) at(6,0.5) {11};
    \node[node] (12) at(8,4) {12};
    \path[edge] (1) -- node[weight] {2} (2);
    \path[edge] (1) -- node[weight] {1} (3);
    \path[edge] (2) -- node[weight] {4} (3);
    \path[edge] (2) -- node[weight] {1} (5);
    \path[edge] (2) -- node[weight] {7} (6);
    \path[edge] (3) -- node[weight] {2} (4);
    \path[edge] (3) -- node[weight] {3} (6);
    \path[edge] (3) -- node[weight,pos=0.25] {1} (8);
    \path[edge] (4) -- node[weight,pos=0.75] {5} (6);
    \path[edge] (4) -- node[weight] {2} (7);
    \path[edge] (4) -- node[weight] {4} (8);
    \path[edge] (5) -- node[weight] {2} (6);
    \path[edge] (5) -- node[weight] {10} (9);
    \path[edge] (6) -- node[weight] {7} (7);
    \path[edge] (6) -- node[weight] {2} (10);
    \path[edge] (7) -- node[weight,pos=0.65] {2} (11);
    \path[edge] (8) -- node[weight] {3} (10);
    \path[edge] (9) -- node[weight] {5} (10);
    \path[edge] (9) -- node[weight] {2} (12);
    \path[edge] (10) -- node[weight] {4} (11);
    \path[edge] (10) -- node[weight] {1} (12);
\end{tikzpicture}\\[0.5cm]


\section*{Aufgabe 19:}
In dieser Aufgabe bezeichnet $e^*$ den Knoten in $L(G)$, der der Kante $e$ in $G$ entspricht.

\begin{compactenum}[(a)]
\item Ein Graph ist genau dann \emph{eulersch}, wenn er zusammenhängend ist und jeder Knoten einen geraden Grad besitzt. Man sieht einfach: $L(G)$ ist zusammenhängend, wenn $G$ zusammenhängend ist. Es bleibt also nur noch zu zeigen, dass alle Knoten von $L(G)$ einen geraden Grad haben, wenn alle Knoten im Originalgraphen $G$ einen geraden Grad haben.\\
Der Grad eines Knoten $e^*$ in $L(G)$, ist genau die Summe der Grade beider Knoten, die $e$ in $G$ verbindet minus $2$. Wenn also die beiden Grade der Knoten gerade sind, ist deren Summe immer noch gerade -- ebenso nach der Subtraktion von $2$.\\
Die Umkehrung gilt nicht, da die Hälfte einer geraden Zahl nicht zwangsweise gerade ist.\\

\item Wenn $G$ eulerisch ist, gibt einen Zyklus, der alle Kanten einmal abläuft. In $L(G)$ entspricht jeder Knoten einer Kante. Weiterhin gilt folgendes: Wenn man in $G$ eine Kante $e_a$ besucht hat und danach eine Kante $e_b$ besuchen \emph{kann} ($e_a$ und $e_b$ haben also einen Endpunkt am selben Knoten), so \emph{kann} man in $L(G)$ nachdem man $e_a^*$ besucht hat, auch den Knoten $e_b^*$ besuchen (eine Kante in $L(G)$ verbindet diese beiden). Dies geht direkt aus der Definition von $L(G)$ hervor. Die Umkehrung gilt ebenfalls.

\end{compactenum}


\end{document}

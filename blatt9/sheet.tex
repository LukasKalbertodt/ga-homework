% Graphenalgorithmen
% Lukas Kalbertodt
% Mirko Wagner
% Elena Resch
% ===== Header ===============================================================
\documentclass[11pt]{scrartcl} % europäische Artikel Klasse
\usepackage[top=3cm, bottom=4.5cm, left=3cm, right=3cm]{geometry}
% Pakete für deutschen Text (Umlaute) + Font
\usepackage[utf8]{inputenc}
\usepackage[T1]{fontenc}
\usepackage{lmodern}
\usepackage{ngerman}
\usepackage[ngerman]{babel} % Deutsche Anführungszeichen
\usepackage{amsmath}
\usepackage{amsfonts} % \mathbb
\usepackage{relsize}
\usepackage{csquotes} % \enquote
\usepackage{paralist} % Beliebige Aufzählungszeichen
\usepackage{tabularx} % Tolle Tabellen
\usepackage{caption}
\usepackage{wasysym}
\usepackage[labelformat=empty]{subcaption}
\usepackage{longtable}
\usepackage{wrapfig}

% Deutsche Absätze: Kein Einrücken, aber Abstand
\parskip=8pt
\parindent=0pt

\usepackage{tikz} % Graphen zeichnen
\usetikzlibrary{arrows}
\usetikzlibrary{positioning}
\usetikzlibrary{decorations.pathmorphing}
\newcommand{\func}[1]{\mbox{\emph{#1}}}

% Dokument-Metadaten
\title{Graphenalgorithmen: Blatt 9}
\author{Lukas Kalbertodt, Elena Resch, Mirko Wagner}

% ===== Body =================================================================
\begin{document}
\maketitle
\definecolor{sblue}{rgb}{0.2,0.2,0.6}
\definecolor{sgreen}{rgb}{0,0.5,0.2}
\definecolor{sred}{rgb}{0.9,0.2,0.2}
\definecolor{syellow}{rgb}{0.9,0.6,0}
\tikzstyle{node} = [circle,draw,inner sep=0pt, minimum size=20pt,font=\scriptsize]
\tikzstyle{oddNode} = [circle,draw,inner sep=0pt, minimum size=20pt,font=\scriptsize,double=black,fill=black!9]
\tikzstyle{edge} = [draw=black,thick]
\tikzstyle{shortEdge} = [ultra thick]
\tikzstyle{weight} = [shape=circle, fill=white,inner sep=1pt,font=\scriptsize]
\section*{Aufgabe 18:}
In dieser Aufgabe wird das Problem des Chinesischen Postboten mit dem beschriebenen Algorithmus gelöst. Einige Zwischenschritte sind mit Grafiken veranschaulicht. Die Färbung dieser Grafiken ist einfach nur, um den Algorithmus besser zu verstehen und folgt manchmal keiner strikten Logik.\\

\textbf{\textsf{\large Schritt 1 \& 2}}\\
Die Knotenmenge $U$ ist im Graph fett und leicht hinterlegt markiert, $|U| = 6$. In Schritt 2 werden die Kürzesten-Wege zwischen den Knoten in $U$ gefunden. Hier im Graph eingezeichnet sind zwei Bäume mit unterschiedlichen Farben. Der kürzeste Weg zwischen zwei Knoten ist exakt der eindeutige Weg in dem Baum mit der Farbe, mit welcher die Kosten in der Kostentabelle stehen. Der Weg von $7$ zu $10$ führt also z.B. über den gelben Weg und nicht über den Blauen.\\

$U = \{3, 5, 7, 8, 9, 10\}$

\vfill
\begin{tabular}{p{0.67\textwidth} p{0.33\textwidth}}
\leavevmode\newline
\begin{tikzpicture}[scale=1.1]
    \node[node] (1) at(0,4) {1};
    \node[node] (2) at(2,5) {2};
    \node[oddNode] (3) at(2,3) {3};
    \node[node] (4) at(2,0.5) {4};
    \node[oddNode] (5) at(4,5) {5};
    \node[node] (6) at(4,3) {6};
    \node[oddNode] (7) at(4,1.5) {7};
    \node[oddNode] (8) at(4,0.5) {8};
    \node[oddNode] (9) at(6,5) {9};
    \node[oddNode] (10) at(6,3) {10};
    \node[node] (11) at(6,0.5) {11};
    \node[node] (12) at(8,4) {12};
    \path[shortEdge, draw=sblue] (1) -- node[weight] {2} (2);
    \path[shortEdge, draw=sblue] (1) -- node[weight] {1} (3);
    \path[edge] (2) -- node[weight] {4} (3);
    \path[shortEdge, draw=sblue] (2) -- node[weight] {1} (5);
    \path[edge] (2) -- node[weight] {7} (6);
    \path[shortEdge, draw=sblue] (3) -- node[weight] {2} (4);
    \path[edge] (3) -- node[weight] {3} (6);
    \path[shortEdge, draw=sblue] (3) -- node[weight,pos=0.25] {1} (8);
    \path[edge] (4) -- node[weight,pos=0.8] {5} (6);
    \path[shortEdge, draw=sblue] (4) -- node[weight] {2} (7);
    \path[edge] (4) -- node[weight] {4} (8);
    \path[shortEdge, draw=syellow] (5) -- node[weight] {2} (6);
    \path[edge] (5) -- node[weight] {10} (9);
    \path[edge] (6) -- node[weight] {7} (7);
    \path[shortEdge, draw=syellow] (6) -- node[weight] {2} (10);
    \path[shortEdge, draw=syellow] (7) -- node[weight,pos=0.65] {2} (11);
    \path[shortEdge, draw=sblue] (8) -- node[weight] {3} (10);
    \path[edge] (9) -- node[weight] {5} (10);
    \path[shortEdge, draw=syellow] (9) -- node[weight] {2} (12);
    \path[shortEdge, draw=sblue] (9) edge [bend angle=15,bend left] (12);
    \path[shortEdge, draw=syellow] (10) -- node[weight] {4} (11);
    \path[shortEdge, draw=syellow] (10) -- node[weight] {1} (12);
    \path[shortEdge, draw=sblue] (10) edge [bend angle=15,bend left] (12);
\end{tikzpicture} &
\newcommand{\cell}[2]{& \textcolor{#1}{\textbf #2}}
\leavevmode\newline
\begin{tabular}{c|c|c|c|c|c|c}
  & 3 & 5 & 7 & 8 & 9 & 10 \\\hline
3 & 0 \cell{sblue}{4} \cell{sblue}{4} \cell{sblue}{1} \cell{sblue}{7} \cell{sblue}{4} \\
5 & - & 0 \cell{sblue}{8} \cell{sblue}{5} \cell{syellow}{7} \cell{syellow}{4}  \\
7 & - & - & 0 \cell{sblue}{5} \cell{syellow}{9} \cell{syellow}{6}\\
8 & - & - & - & 0  \cell{sblue}{6}  \cell{sblue}{3} \\
9 & - & - & - & - & 0 \cell{sblue}{3} \\
10 & - & - & - & - & - & 0 \\
\end{tabular}
\end{tabular}

\newpage
\textbf{\textsf{\large Schritt 3}}\\
\begin{tabular}{p{0.5\textwidth} p{0.5\textwidth}}
\leavevmode\newline
\begin{tikzpicture}[scale=1.1]
    \node[node] (3) at(-2.5,0) {3};
    \node[node] (5) at(-1.25,2.165) {5};
    \node[node] (7) at(-1.25,-2.165) {7};
    \node[node] (8) at(1.25,-2.165) {8};
    \node[node] (9) at(1.25,2.165) {9};
    \node[node] (10) at(2.5,0) {10};

    \path[thin,draw=sblue!70] (3) -- node[weight] {4} (5);
    \path[ultra thick,draw=sblue] (3) -- node[weight] {4} (7);
    \path[thin,draw=sblue!70] (3) -- node[weight,pos=0.4] {1} (8);
    \path[thin,draw=sblue!70] (3) -- node[weight,pos=0.4] {7} (9);
    \path[thin,draw=sblue!70] (3) -- node[weight,pos=0.35] {4} (10);

    \path[thin,draw=sblue!70] (5) -- node[weight,pos=0.4] {8} (7);
    \path[thin,draw=sblue!70] (5) -- node[weight,pos=0.35] {5} (8);
    \path[ultra thick,draw=syellow] (5) -- node[weight] {7} (9);
    \path[thin,draw=syellow!70] (5) -- node[weight,pos=0.4] {4} (10);

    \path[thin,draw=sblue!70] (7) -- node[weight] {5} (8);
    \path[thin,draw=syellow!70] (7) -- node[weight,pos=0.35] {9} (9);
    \path[thin,draw=syellow!70] (7) -- node[weight,pos=0.4] {6} (10);

    \path[thin,draw=sblue!70] (8) -- node[weight,pos=0.4] {6} (9);
    \path[ultra thick,draw=sblue] (8) -- node[weight] {3} (10);

    \path[thin,draw=sblue!70] (9) -- node[weight] {6} (10);
\end{tikzpicture} &
In Schritt 3 wird auf dem vollständigen Graphen über $U$ mit den Kantengewichten aus dem vorherigen Schritt ein minimales, perfektes Matching gebildet. Auf der linken Seite sieht man den vollständigen Graphen und ein mögliches Matching eingezeichnet. Der Wert des Matchings ist:\newline\newline
$|M| = 14$
\end{tabular}\\[0.4cm]
\textbf{\textsf{\large Schritt 4}}\\
\begin{wrapfigure}{r}{0.55\textwidth}
\begin{tikzpicture}[-,scale=1.1]
    \node[node] (1) at(0,4) {1};
    \node[node] (2) at(2,5) {2};
    \node[node] (3) at(2,3) {3};
    \node[node] (4) at(2,0.5) {4};
    \node[node] (5) at(4,5) {5};
    \node[node] (6) at(4,3) {6};
    \node[node] (7) at(4,1.5) {7};
    \node[node] (8) at(4,0.5) {8};
    \node[node] (9) at(6,5) {9};
    \node[node] (10) at(6,3) {10};
    \node[node] (11) at(6,0.5) {11};
    \node[node] (12) at(8,4) {12};
    \path[edge] (1) -- node[weight] {2} (2);
    \path[edge] (1) -- node[weight] {1} (3);
    \path[edge] (2) -- node[weight] {4} (3);
    \path[edge] (2) -- node[weight] {1} (5);
    \path[edge] (2) -- node[weight] {7} (6);
    \path[edge] (3) -- node[weight] {2} (4);
    \path[edge] (3) -- node[weight] {3} (6);
    \path[ultra thick,draw=sblue] (3) -- node[weight,pos=0.5] {4} (7);
    \path[edge] (3) -- node[weight,pos=0.25] {1} (8);
    \path[edge] (4) -- node[weight,pos=0.8] {5} (6);
    \path[edge] (4) -- node[weight] {2} (7);
    \path[edge] (4) -- node[weight] {4} (8);
    \path[edge] (5) -- node[weight] {2} (6);
    \path[edge] (5) -- node[weight] {10} (9);
    \path[ultra thick, draw=syellow] (5) edge[bend right] node[weight] {7} (9);
    \path[edge] (6) -- node[weight] {7} (7);
    \path[edge] (6) -- node[weight] {2} (10);
    \path[edge] (7) -- node[weight,pos=0.65] {2} (11);
    \path[edge] (8) -- node[weight] {3} (10);
    \path[ultra thick,draw=sblue] (8) edge[bend angle=20,bend right] node[weight] {3} (10);
    \path[edge] (9) -- node[weight] {5} (10);
    \path[edge] (9) -- node[weight] {2} (12);
    \path[edge] (10) -- node[weight] {4} (11);
    \path[edge] (10) -- node[weight] {1} (12);
\end{tikzpicture}
\end{wrapfigure}
Die in Schritt 3 berechneten Kanten des minimalen, perfekten Matchings fügen wir zum eigentlichen Graphen hinzu, sodass aus diesem ein Multigraph werden kann. Die 3 hinzugefügten Kanten sind in der Grafik rechts markiert.

Auf diesem Multigraphen berechnen wir nun eine beliebige Eulertour (siehe Kapitel 1.4 im Script). In der folgenden Grafik ist diese Eulertour farblich markiert (Schwarz $\rightarrow$ Grün $\rightarrow$ Blau $\rightarrow$ Rot).

Die Länge der Postbotentour entspricht der Summe aller Kantengewichte im originalen Graphen plus die Gewichte der Matching-Kanten, die hinzugefügt wurden (14), also $\implies$  \textbf{84}.
\begin{center}
\begin{tikzpicture}[->,scale=1.13]
    \node[node] (1) at(0,4) {1};
    \node[node] (2) at(2,5) {2};
    \node[node] (3) at(2,3) {3};
    \node[node] (4) at(2,0.5) {4};
    \node[node] (5) at(4,5) {5};
    \node[node] (6) at(4,3) {6};
    \node[node] (7) at(4,1.5) {7};
    \node[node] (8) at(4,0.5) {8};
    \node[node] (9) at(6,5) {9};
    \node[node] (10) at(6,3) {10};
    \node[node] (11) at(6,0.5) {11};
    \node[node] (12) at(8,4) {12};

    \path[edge,draw=green!0!black] (1) -- node[weight] {2} (2);
    \path[edge,draw=green!15!black] (2) -- node[weight] {1} (5);
    \path[edge,draw=green!30!black] (5) -- node[weight] {10} (9);
    \path[edge,draw=green!45!black] (9) -- node[weight] {2} (12);
    \path[edge,draw=green!60!black] (12) -- node[weight] {1} (10);
    \path[edge,draw=green!75!black] (10) -- node[weight] {5} (9);
    \path[edge,draw=green!90!black] (9) edge[bend left] node[weight] {7} (5);

    \path[edge,draw=blue!5!green] (5) -- node[weight] {2} (6);
    \path[edge,draw=blue!20!green] (6) -- node[weight] {7} (2);
    \path[edge,draw=blue!35!green] (2) -- node[weight] {4} (3);
    \path[edge,draw=blue!50!green] (3) -- node[weight] {3} (6);
    \path[edge,draw=blue!65!green] (6) -- node[weight] {2} (10);
    \path[edge,draw=blue!80!green] (10) -- node[weight] {4} (11);
    \path[edge,draw=blue!95!green] (11) -- node[weight,pos=0.35] {2} (7);

    \path[edge,draw=red!10!blue] (7) -- node[weight] {7} (6);
    \path[edge,draw=red!25!blue] (6) -- node[weight,pos=0.7] {5} (4);
    \path[edge,draw=red!40!blue] (4) -- node[weight] {2} (7);
    \path[edge,draw=red!55!blue] (7) -- node[weight,pos=0.5] {4} (3);
    \path[edge,draw=red!70!blue] (3) -- node[weight] {2} (4);
    \path[edge,draw=red!85!blue] (4) -- node[weight] {4} (8);
    \path[edge,draw=red!100!blue] (8) -- node[weight] {3} (10);
    \path[edge,draw=black!30!red] (10) edge[bend angle=20,bend left] node[weight] {3} (8);
    \path[edge,draw=black!50!red] (8) -- node[weight,pos=0.75] {1} (3);
    \path[edge,draw=black!90!red] (3) -- node[weight] {1} (1);

\end{tikzpicture}
\end{center}

\newpage
\section*{Aufgabe 19:}
In dieser Aufgabe bezeichnet $e^*$ den Knoten in $L(G)$, der der Kante $e$ in $G$ entspricht.

\begin{compactenum}[(a)]
\item Ein Graph ist genau dann \emph{eulersch}, wenn er zusammenhängend ist und jeder Knoten einen geraden Grad besitzt. Man sieht einfach: $L(G)$ ist zusammenhängend, wenn $G$ zusammenhängend ist. Es bleibt also nur noch zu zeigen, dass alle Knoten von $L(G)$ einen geraden Grad haben, wenn alle Knoten im Originalgraphen $G$ einen geraden Grad haben.\\
Der Grad eines Knoten $e^*$ in $L(G)$, ist genau die Summe der Grade beider Knoten, die $e$ in $G$ verbindet minus $2$. Wenn also die beiden Grade der Knoten gerade sind, ist deren Summe immer noch gerade -- ebenso nach der Subtraktion von $2$.\\
Die Umkehrung gilt nicht, da die Hälfte einer geraden Zahl nicht zwangsweise gerade ist.\\

\item Wenn $G$ eulerisch ist, gibt einen Zyklus, der alle Kanten einmal abläuft. In $L(G)$ entspricht jeder Knoten einer Kante. Weiterhin gilt folgendes: Wenn man in $G$ eine Kante $e_a$ besucht hat und danach eine Kante $e_b$ besuchen \emph{kann} ($e_a$ und $e_b$ haben also einen Endpunkt am selben Knoten), so \emph{kann} man in $L(G)$ nachdem man $e_a^*$ besucht hat, auch den Knoten $e_b^*$ besuchen (eine Kante in $L(G)$ verbindet diese beiden). Dies geht direkt aus der Definition von $L(G)$ hervor. Die Umkehrung gilt ebenfalls.

\end{compactenum}


\end{document}

% Graphenalgorithmen
% Lukas Kalbertodt
% Mirco Wagner
% Elena Resch
% ===== Header ===============================================================
\documentclass[11pt]{scrartcl} % europäische Artikel Klasse
\usepackage[top=3cm, bottom=4.5cm, left=3.25cm, right=3.25cm]{geometry}
% Pakete für deutschen Text (Umlaute) + Font
\usepackage[utf8]{inputenc}
\usepackage[T1]{fontenc}
\usepackage{lmodern}
\usepackage{ngerman}
\usepackage[ngerman]{babel} % Deutsche Anführungszeichen
\usepackage{amsmath}
\usepackage{amsfonts} % \mathbb
\usepackage{relsize}
\usepackage{csquotes} % \enquote
\usepackage{paralist} % Beliebige Aufzählungszeichen
\usepackage{tabularx} % Tolle Tabellen
\usepackage{caption}
\usepackage[labelformat=empty]{subcaption}

\usepackage{tikz} % Graphen zeichnen
\usetikzlibrary{arrows}
\usetikzlibrary{positioning}
\newcommand{\func}[1]{\mbox{\emph{#1}}}

% Dokument-Metadaten
\title{Graphenalgorithmen: Blatt 6}
\author{Lukas Kalbertodt, Elena Resch, Mirco Wagner}

% ===== Body =================================================================
\begin{document}
\maketitle


\section*{Aufgabe 13:}
\tikzstyle{edge} = [draw=black,thick]
\tikzstyle{weight} = [shape=circle, fill=white,inner sep=1pt,font=\small]
\begin{compactenum}[(a)]
\item Der Kürzeste-Wege-Baum ist nicht eindeutig bestimmt. Die Kantenkosten sind zwar alle unterschiedlich, aber d.h. nicht, dass es nicht zwei Wege zum selben Knoten mit den selben Kosten geben kann. Beispiel:\\
    \begin{center}
    \begin{tikzpicture}[->,node distance=2.5cm,node/.style={circle,draw,inner sep=0pt, minimum size=20pt}]
        \node[node] (a) {s};
        \node[node] (b) [below left of=a] {};
        \node[node] (c) [below right of=a] {};
        \path[edge] (a) -- node[weight] {1} (b);
        \path[edge] (a) -- node[weight] {4} (c);
        \path[edge] (b) -- node[weight] {3} (c);
    \end{tikzpicture}\\[0.5cm]
    \end{center}

\item Offentsichtlich ist erstmal jeder Kürzeste-Wege-Baum ein Spannbaum. Dieser Spannbaum ist aber nicht immer minimal. Im folgenden Beispiel sind die markierten Kanten zwar ein gültiger Kürzeste-Wege-Baum, aber kein minimaler Spannbaum:\\
    \begin{center}
    \begin{tikzpicture}[-,node distance=2.5cm,node/.style={circle,draw,inner sep=0pt, minimum size=20pt}]
        \node[node] (a) {s};
        \node[node] (b) [below left of=a] {};
        \node[node] (c) [below right of=a] {};
        \path[edge,ultra thick] (a) -- node[weight] {1} (b);
        \path[edge,ultra thick] (a) -- node[weight] {4} (c);
        \path[edge] (b) -- node[weight] {3} (c);
    \end{tikzpicture}\\[0.5cm]
    \end{center}
\end{compactenum}
\end{document}

% \tikzstyle{weight} = [draw=none,shape=circle, fill=white,inner sep=1pt,font=\scriptsize]

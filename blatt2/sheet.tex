% Graphenalgorithmen
% Lukas Kalbertodt
% Mirco Wagner
% Elena Resch

% ===== Header ===============================================================
\documentclass[11pt]{scrartcl}  % europäische Artikel Klasse
\usepackage[top=3cm, bottom=4.5cm, left=3.25cm, right=3.25cm]{geometry}

% Pakete für deutschen Text (Umlaute) + Font
\usepackage[utf8]{inputenc}
\usepackage[T1]{fontenc}
\usepackage{lmodern}
\usepackage{ngerman}
\usepackage[ngerman]{babel} % Deutsche Anführungszeichen
\usepackage{amsmath}
\usepackage{amsfonts}   % \mathbb
\usepackage{relsize}
\usepackage{csquotes}   % \enquote
\usepackage{paralist}   % Beliebige Aufzählungszeichen
\usepackage{tabularx}   % Tolle Tabellen
\usepackage{caption}
\usepackage[labelformat=empty]{subcaption}

\usepackage{tikz} % Graphen zeichnen
\usetikzlibrary{arrows}

\newcommand{\func}[1]{\mbox{\emph{#1}}}

% commands for sub task (a) ...
\newcommand{\subtbegin}{\begin{compactenum}[(a)]}
\newcommand{\subtend}{\end{compactenum}}

% Dokument-Metadaten
\title{Graphenalgorithmen: Blatt 2}
\author{Lukas Kalbertodt, Elena Resch, Mirco Wagner}

% ===== Body =================================================================
\begin{document}
\maketitle


\section*{Aufgabe 4:}

\begin{compactenum}[(a)]
\item
  \begin{compactenum}[(i)]
    \item Ist zyklisch (\texttt{A -> C -> B}) und stark zusammenhängend.
    \item Ist \emph{weder} zyklisch \emph{noch} stark zusammenhängend, aber zusammenhängend.
    \item Hä?
  \end{compactenum}
  \vspace{0.4cm}

  \begin{figure}[h]
    \centering
    \begin{subfigure}{0.3\textwidth}
      \centering
      \begin{tikzpicture}[->,>=stealth',auto,node distance=2cm, thick,main node/.style={circle,draw}]
        \node[main node] (A) {A};
        \node[main node] (B) [below of=A] {B};
        \node[main node] (C) [right of=A] {C};
        \node[main node] (D) [right of=B] {D};
        \path[]
          (A) edge node {} (B)
              edge node {} (C)
          (B) edge node {} (A)
          (C) edge node {} (B)
              edge node {} (D)
          (D) edge node {} (B);
      \end{tikzpicture}

      \caption{(i)}
    \end{subfigure}
    \begin{subfigure}{0.3\textwidth}
      \centering
      \begin{tikzpicture}[->,>=stealth',auto,node distance=2cm, thick,main node/.style={circle,draw}]
        \node[main node] (A) {A};
        \node[main node] (B) [below of=A] {B};
        \node[main node] (C) [right of=A] {C};
        \node[main node] (D) [right of=B] {D};
        \path[]
          (A) edge node {} (B)
          (B) edge node {} (C)
              edge node {} (D)
          (C) edge node {} (D);
      \end{tikzpicture}

      \caption{(ii)}
    \end{subfigure}
    \begin{subfigure}{0.3\textwidth}
      \centering
      \begin{tikzpicture}[->,>=stealth',auto,node distance=2cm, thick,main node/.style={circle,draw}]
        \node[main node] (A) {A};
        \node[main node] (B) [below of=A] {B};
        \node[main node] (C) [right of=A] {C};
        \node[main node] (D) [right of=B] {D};
        \path[]
          (A) edge node {} (B)
          (B) edge node {} (C)
              edge node {} (D)
          (C) edge node {} (D);
      \end{tikzpicture}

      \caption{(iii)}
    \end{subfigure}
  \end{figure}

\item Blaaa

\end{compactenum}

\end{document}

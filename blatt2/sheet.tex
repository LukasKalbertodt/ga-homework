% Graphenalgorithmen
% Lukas Kalbertodt
% Mirco Wagner
% Elena Resch
% ===== Header ===============================================================
\documentclass[11pt]{scrartcl} % europäische Artikel Klasse
\usepackage[top=3cm, bottom=4.5cm, left=3.25cm, right=3.25cm]{geometry}
% Pakete für deutschen Text (Umlaute) + Font
\usepackage[utf8]{inputenc}
\usepackage[T1]{fontenc}
\usepackage{lmodern}
\usepackage{ngerman}
\usepackage[ngerman]{babel} % Deutsche Anführungszeichen
\usepackage{amsmath}
\usepackage{amsfonts} % \mathbb
\usepackage{relsize}
\usepackage{csquotes} % \enquote
\usepackage{paralist} % Beliebige Aufzählungszeichen
\usepackage{tabularx} % Tolle Tabellen
\usepackage{caption}
\usepackage[labelformat=empty]{subcaption}
\usepackage{tikz} % Graphen zeichnen
\usetikzlibrary{arrows}
\usetikzlibrary{positioning}
\newcommand{\func}[1]{\mbox{\emph{#1}}}
% commands for sub task (a) ...
\newcommand{\subtbegin}{\begin{compactenum}[(a)]}
\newcommand{\subtend}{\end{compactenum}}
% Dokument-Metadaten
\title{Graphenalgorithmen: Blatt 2}
\author{Lukas Kalbertodt, Elena Resch, Mirco Wagner}
% ===== Body =================================================================
\begin{document}
\maketitle
\section*{Aufgabe 4:}
\begin{compactenum}[(a)]
\item
\begin{compactenum}[(i)]
\item Ist zyklisch (\texttt{A -> C -> B}) und stark zusammenhängend.
\item Ist \emph{weder} zyklisch \emph{noch} stark zusammenhängend, aber zusammenhängend.
\item Ist zyklisch (\texttt{A -> C -> B}), aber nicht zusammenhängend (also auch nicht stark zusammenhängend).
\end{compactenum}
\vspace{0.4cm}
\begin{figure}[h]
\centering
\begin{subfigure}{0.3\textwidth}
\centering
\begin{tikzpicture}[->,>=stealth',auto,node distance=2cm, thick,main node/.style={circle,draw}]
\node[main node] (A) {A};
\node[main node] (B) [below of=A] {B};
\node[main node] (C) [right of=A] {C};
\node[main node] (D) [right of=B] {D};
\path[]
(A) edge node {} (B)
edge node {} (C)
(B) edge node {} (A)
(C) edge node {} (B)
edge node {} (D)
(D) edge node {} (B);
\end{tikzpicture}
\caption{(i)}
\end{subfigure}
\begin{subfigure}{0.3\textwidth}
\centering
\begin{tikzpicture}[->,>=stealth',auto,node distance=2cm, thick,main node/.style={circle,draw}]
\node[main node] (A) {A};
\node[main node] (B) [below of=A] {B};
\node[main node] (C) [right of=A] {C};
\node[main node] (D) [right of=B] {D};
\path[]
(A) edge node {} (B)
(B) edge node {} (C)
edge node {} (D)
(C) edge node {} (D);
\end{tikzpicture}
\caption{(ii)}
\end{subfigure}
\begin{subfigure}{0.3\textwidth}
\centering
\begin{tikzpicture}[->,>=stealth',auto,node distance=1.5cm, thick,main node/.style={circle,draw}]
\node[main node] (A) {A};
\node[main node] (B) [below left of=A] {B};
\node[main node] (C) [below right of=A] {C};
\node[main node] (D) [below=0.2cm of B] {D};
\node[main node] (E) [below=0.2cm of C] {E};
\path[]
(A) edge node {} (C)
(B) edge node {} (A)
(C) edge node {} (B)
(D) edge node {} (E);
\end{tikzpicture}
\caption{(iii)}
\end{subfigure}
\end{figure}
\item In der $n\times n$ Matrix $B$ mit ist $b_{ii} = \sum_{j=1}^{n} A_{ij} \cdot A_{ji}$. Da $A$ ungerichtet ist, ist $A_{ij} = A_{ji}$. Der Eintrag $A_{ij}$ hat genau dann eine $1$, wenn eine Kante von $i$ nach $j$ existiert. Weil $1^2 = 1$ und $0^2 = 0$, sind alle Summanden entweder $0$ oder $1$. Die Einsen (d.h. die Kanten von Knoten $i$) werden aufaddiert, also gezählt, und somit ergibt der Eintrag $b_{ii}$den Knotengrad von $i$.
\item Wenn sich zwei Knoten $e_1$ und $e_2$ in der selben starken Zusammenhangskomponente befindet, gibt es einen gerichteten Weg von $e_1$ zu $e_2$ und umgekehrt. Das heißt, dass in dem transitiven Abschluss des Graphen, eine direkte Kante von $e_1$ nach $e_2$ und von $e_2$ nach $e_1$ existiert.
Die Formel zur Berrechnung von $b_{ii}$ aus (b) lautet: $b_{ii} = \sum_{j=1}^{n} A_{ij} \cdot A_{ji}$. Allerdings ist der transitive Abschluss gerichtet, d.h. $A_{ij}$ und $A_{ji}$ müssen nicht gleich sein. Ein Summand ist nur genau dann $1$, wenn es eine Kante von $i$ nach $j$ und umgekehrt gibt. Wie oben bereits gesagt ist das genau dann der Fall, wenn $i$ und $j$ im Ursprungsgraphen in der selben starken Zusammenhangskomponente waren. Mit der Summe zählen wir die Einsen, also die Anzahl der Knoten in der selben starken Zusammenhangskomponente.
\end{compactenum}
\newpage
\section*{Aufgabe 5:}
\begin{compactenum}[(a)]
\item
\emph{Adjazenzmatrix}: Es muss ein Element der Matrix überprüft werden. $\rightarrow \mathcal O(1)$\\[0.1cm]
\emph{Adjazentlisten}: Es wird bei dem kleineren Knoten seine Adjazentliste nach dem größeren Knoten durchsucht. Im Worstcase könnte die Liste $|V|$ Einträge halten, die man alle nach dem größeren Knoten durchsuchen muss. $\rightarrow \mathcal O(|V|)$\\
\item
Da jeder Knoten $n$ Nachbarn haben könnte, dauert es also mindestens $\Omega(|V|)$.\\[0.2cm]
\emph{Adjazenzmatrix}: Es muss eine komplette Spalte oder Zeile (in $\mathcal O$ Notation gleichwertig) durchgegangen werden. Auch wenn der Knoten nur wenige oder keine Nachbarn hat, muss man trotzdem immer alle $\mathcal O(|V|)$ Einträge durchsuchen. $\rightarrow \mathcal O(|V|)$\\[0.1cm]
\emph{Adjazentlisten}: Man fügt zuerst alle Einträge in der Adjazentliste des betreffenden Knoten zu seiner Ergebnismenge hinzu. Danach muss in den Listen aller kleinerer Knoten ($\mathcal O(|V|)$) nach dem betreffenden Knoten gesucht werden ($O(|V|)$). $\rightarrow \mathcal O(|V| ^2)$
\end{compactenum}
\end{document}


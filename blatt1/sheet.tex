% Graphenalgorithmen
% Lukas Kalbertodt
% Mirco Wagner
% Elena Resch

% ===== Header ===============================================================
\documentclass[11pt]{scrartcl}  % europäische Artikel Klasse
\usepackage[top=3cm, bottom=4.5cm, left=3.25cm, right=3.25cm]{geometry}

% Pakete für deutschen Text (Umlaute) + Font
\usepackage[utf8]{inputenc}
\usepackage[T1]{fontenc}
\usepackage{lmodern}
\usepackage{ngerman}
\usepackage[ngerman]{babel} % Deutsche Anführungszeichen
\usepackage{amsmath}
\usepackage{amsfonts}   % \mathbb
\usepackage{relsize}
\usepackage{csquotes}   % \enquote
\usepackage{paralist}   % Beliebige Aufzählungszeichen

\usepackage{tikz} % Graphen zeichnen
\usetikzlibrary{arrows}

\newcommand{\func}[1]{\mbox{\emph{#1}}}

% commands for sub task (a) ...
\newcommand{\subtbegin}{\begin{compactenum}[(a)]}
\newcommand{\subtend}{\end{compactenum}}

% Dokument-Metadaten
\title{Graphenalgorithmen: Blatt 1}
\author{Lukas Kalbertodt, Elena Resch, Mirco Wagner}

% ===== Body =================================================================
\begin{document}
\maketitle


\section*{Aufgabe 1:}

\subtbegin
  \item Bei dem Graphen mit $n$ Knoten, kann ein gewählter Knoten nur folgende Knotengrade besitzen: $0, 1, \dots (n-1)$. Dies gilt zwar für jeden Knoten, allerdings gilt folgende Einschränkung: Wenn ein Knoten den Knotengrad $n-1$ besitzt, kann es keinen Knoten mehr mit dem Knotengrad $0$ geben. Insgesamt können also nur $n-1$ Knotengrade in dem Graphen existieren. Bei $n$ Knoten heißt das, dass mindestens zwei Knoten einen doppelten Grad haben.
  \item Informale Induktion: Mit zwei Knoten (also einer Kante) ist es offentsichtlich, dass beide Knoten den Grad $1$ haben. Wenn wir einen Knoten und eine Kante hinzufügen, erhöhen wir den Knotengrad von nur einem Knoten (nur ein Knoten mit Grad $1$ könnte verloren gehen). Allerdings fügen wir auch einen weiteren Knoten mit Grad $1$ hinzu, daher kann sich die Anzahl der Knoten mit Grad $1$ nie verringern. Ist halt ein Pfad \dots
\subtend

\section*{Aufgabe 2:}

Der Sonderfall $n=1$, bei dem man nur eine Farbe braucht, wird im Folgenden nicht betrachtet.

\begin{itemize}
  \item $\chi (K_n) = n$ \\
  In einem vollständigen Graphen sind alle Knoten miteinander verbunden, d.h. alle Knoten müssen eine andere Farbe haben.
  \item $\chi (K_{m, n}) = 2$ \\
  Jeder Knoten hat nur Nachbarn aus der anderen Partition. Wenn man jede Partition mit einer Farbe einfärbt, reicht das schon aus.
  \item $\chi (P_n) = 2$ \\
  Man kann alle Knoten entlang des Pfades in abwechselnder Farbe einfärben. So ist der Vorgänger und Nachfolger eines Knoten immer in einer anderen Farbe eingefärbt.
  \item $\chi (C_n) =
    \begin{cases}
      2 & n \text{ gerade} \\
      3 & n \text{ ungerade}
    \end{cases}$ \\
  Man färbt wie beim Pfad ein. Wenn $n$ allerdings ungerade ist, sind an einer Stelle zwei selbe Farben benachbart. Dort färbt man dann einen der beiden Knoten in einer dritten Farbe.
  \item $\chi (\text{bipartiter Graph}) = 2$ \\
  Wie beim vollständigen bipartiten Graphen. Falls der Graph nicht vollständig ist, gibt es ja nur weniger Kanten $\rightarrow$ weniger Nachbarn. Also wird das Problem nur erleichtert. Und solange es mindestens eine Kante gibt, braucht man 2 Farben.
  \item $\chi '(C_n) =
    \begin{cases}
      2 & n \text{ gerade} \\
      3 & n \text{ ungerade}
    \end{cases}$ \\
    Selbe Begründung wie bei $\chi (C_n)$.
\end{itemize}


\section*{Aufgabe 3:}

\subtbegin
\item Jeder Knoten enthällt folgende Daten: $(\text{Nächster Spieler}, \text{Steine auf dem Tisch})$. Dabei steht $A$ für den Startspieler und $B$ für den Spieler, der als zweites dran ist.\\

\begin{center}
\begin{tikzpicture}[->,>=stealth',shorten >=1pt,auto,node distance=3.5cm,
  thick,main node/.style={circle,draw,font=\large}]

  \node[main node, double=black] (A5) {($A, 5$)};

  \node[main node] (B4) [left of=A5] {($B, 4$)};
  \node[main node] (A2) [below of=A5] {($A, 2$)};
  \node[main node] (B3) [right of=A2] {($B, 3$)};
  \node[main node] (A1) [below of=B3] {($A, 1$)};
  \node[main node] (B1) [below of=B4] {($B, 1$)};
  \node[main node] (A3) [left of=B1] {($A, 3$)};
  \node[main node] (B2) [below of=A3] {($B, 2$)};
  \node[main node, black!60!green!80] (B0) [below of=A2] {($B, 0$)};
  \node[main node, black!60!green!80] (A0) [below of=B1] {($A, 0$)};

  \path[every node/.style={font=\sffamily\tiny}]
    (A5) edge [] node[] {A nimmt 1} (B4)
         edge [] node[] {A nimmt 2} (B3)
    (A3) edge [] node[] {A nimmt 1} (B2)
         edge [] node[] {A nimmt 2} (B1)
    (A2) edge [] node[] {A nimmt 2} (B0)
         edge [] node[] {A nimmt 1} (B1)
    (A1) edge [] node[] {A nimmt 1} (B0)

    (B4) edge [] node[] {B nimmt 1} (A3)
         edge [] node[] {B nimmt 2} (A2)
    (B3) edge [] node[] {B nimmt 1} (A2)
         edge [] node[] {B nimmt 2} (A1)
    (B2) edge [bend right] node[] {B nimmt 1} (A1)
         edge [] node[] {B nimmt 2} (A0)
    (B1) edge [] node[] {B nimmt 1} (A0);
\end{tikzpicture}
\end{center}

\item Es gibt sieben unterschiedliche Spielverläufe, bei drei davon gewinnt $A$, bei vier davon gewinnt $B$.\\
\item Spieler $A$ kann immer gewinnen, indem er im ersten Zug nur einen Stein nimmt [$\rightarrow (B, 4)$]. Danach muss er abhängig von $B$'s Zug so viel Steine nehmen, dass das Spiel auf $(B, 1)$ landet.\\[0.15cm]
Spieler $B$ kann entsprechend nicht immer gewinnen.

\subtend

\end{document}

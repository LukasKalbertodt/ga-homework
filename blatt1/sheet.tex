% Graphenalgorithmen
% Lukas Kalbertodt
%

% ===== Header ===============================================================
\documentclass[11pt]{scrartcl}  % europäische Artikel Klasse

% Pakete für deutschen Text (Umlaute) + Font
\usepackage[utf8]{inputenc}
\usepackage[T1]{fontenc}
\usepackage{lmodern}
\usepackage{ngerman}
\usepackage[ngerman]{babel} % Deutsche Anführungszeichen
\usepackage{amsmath}
\usepackage{amsfonts}   % \mathbb
\usepackage{relsize}
\usepackage{csquotes}   % \enquote
\usepackage{paralist}   % Beliebige Aufzählungszeichen

\newcommand{\func}[1]{\mbox{\emph{#1}}}

% commands for sub task (a) ...
\newcommand{\subtbegin}{\begin{compactenum}[(a)]}
\newcommand{\subtend}{\end{compactenum}}

% Dokument-Metadaten
\title{Graphenalgorithmen: Blatt 1}
\author{Lukas Kalbertodt}

% ===== Body =================================================================
\begin{document}
\maketitle


\section*{Aufgabe 1:}

\subtbegin
  \item Bei dem Graphen mit $n$ Knoten, kann ein gewählter Knoten nur folgende Knotengrade besitzen: $0, 1, \dots (n-1)$. Dies gilt zwar für jeden Knoten, allerdings gilt folgende Einschränkung: Wenn ein Knoten den Knotengrad $n-1$ besitzt, kann es keinen Knoten mehr mit dem Knotengrad $0$ geben. Insgesamt können also nur $n-1$ Knotengrade in dem Graphen existieren. Bei $n$ Knoten heißt das, dass mindestens zwei Knoten einen doppelten Grad haben.
  \item Informale Induktion: Mit zwei Knoten (also einer Kante) ist es offentsichtlich, dass beide Knoten den Grad $1$ haben. Wenn wir einen Knoten und eine Kante hinzufügen, erhöhen wir den Knotengrad von nur einem Knoten (nur ein Knoten mit Grad $1$ könnte verloren gehen). Allerdings fügen wir auch einen weiteren Knoten mit Grad $1$ hinzu, daher kann sich die Anzahl der Knoten mit Grad $1$ nie verringern. Ist halt ein Pfad \dots
\subtend

\end{document}

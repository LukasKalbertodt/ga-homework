% Graphenalgorithmen
% Lukas Kalbertodt
% Mirko Wagner
% Elena Resch
% ===== Header ===============================================================
\documentclass[11pt]{scrartcl} % europäische Artikel Klasse
\usepackage[top=3cm, bottom=4.5cm, left=3.25cm, right=3.25cm]{geometry}
% Pakete für deutschen Text (Umlaute) + Font
\usepackage[utf8]{inputenc}
\usepackage[T1]{fontenc}
\usepackage{lmodern}
\usepackage{ngerman}
\usepackage[ngerman]{babel} % Deutsche Anführungszeichen
\usepackage{amsmath}
\usepackage{amsfonts} % \mathbb
\usepackage{relsize}
\usepackage{csquotes} % \enquote
\usepackage{paralist} % Beliebige Aufzählungszeichen
\usepackage{tabularx} % Tolle Tabellen
\usepackage{caption}
\usepackage[labelformat=empty]{subcaption}

\usepackage{tikz} % Graphen zeichnen
\usetikzlibrary{arrows}
\usetikzlibrary{positioning}
\newcommand{\func}[1]{\mbox{\emph{#1}}}

% Dokument-Metadaten
\title{Graphenalgorithmen: Blatt 7}
\author{Lukas Kalbertodt, Elena Resch, Mirko Wagner}

% ===== Body =================================================================
\begin{document}
\maketitle


\section*{Aufgabe 14:}

Im Algorithmus von Dijkstra wird für jede ausgehende Kante eines Knoten folgendes ausgeführt (vgl. Script Abbildung 4.2, Zeile 7):
\[d[j] = \text{min}\{d[j], d[i] + c_{ij}\}\]
Um die Wege maximaler Kapazitäten rauszufinden, wird dieses Update geändert in:
\[d[j] = \text{max}\{d[j], \text{min}\{d[i], c_{ij}\}\}\]
Außerdem müssen wir bei Bestimmung des nächsten Knotens (vgl. Script Abbildung 4.2, Zeile 4) nicht den Knoten mit minimalem $d[i]$ wählen, sondern mit dem maximalen $d[i]$. So garantieren wir, dass wenn ein Knoten gewählt wird, sein Weg-Wert schon maximal ist und sich nicht mehr verändern wird.


\end{document}
